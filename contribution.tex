\documentclass{article}
\usepackage[style=apa, backend=biber]{biblatex}
\addbibresource{contribution.bib}
\usepackage{graphicx}
\usepackage{booktabs}
\usepackage{indentfirst}
\usepackage{amsmath}
\usepackage{setspace}
\usepackage[margin=1in]{geometry}
\usepackage[table,x11names]{xcolor}
\usepackage{fancyhdr}
\pagestyle{fancy}
\fancyhf{}
\renewcommand{\headrulewidth}{0pt}
\fancyhead[R]{\thepage}
\usepackage{titlesec}
\titleformat{\section}[block]{\normalsize\bfseries\filcenter}{}{1em}{}

\titleformat{\subsection}[block]{\normalsize\bfseries\filcenter}{}{1em}{\itshape}
\doublespacing

\begin{document}
\begin{titlepage}
	\centering
	\vspace*{0.2\textheight}
	\textbf{An expanded glance at environmental influences on ejectives in world languages}\\
	Lyndsay Ricks\\
	Department of Linguistics, University of Utah\\
	WRTG3012: Writing in the Social Sciences\\
	Nona Brown\\
	December 11, 2022
\end{titlepage}
\section{Abstract}

{\parindent0pt Previous studies have reported an association between the likelihood of a given language having phonemic ejective consonants and the elevation at which it is spoken, but have omitted consideration of additional climatic factors. This study expands on existing research by measuring and controlling for three key climate features associated with elevation (average annual temperature, temperature range, and precipitation), and by using measures of not only the presence of ejectives, but also their prevalence. The results of this analysis indicated that previous relationships identified between elevation and the prevalence of ejectives are potentially accurate, even when accounting for inheritance and additional environmental factors. Precipitation is also shown to be a potential mitigating factor in the effect of elevation on the prevalence of ejectives. Implications for future linguistic typological research are discussed.} \\

{\parindent0pt Keywords: \emph{phonology, climate, linguistic typology, ejectives, elevation, climate, linguistic geography, ecolinguistics}}

\pagebreak

\section{Introduction}
In recent years, rapid technological development has led to the proliferation of advanced databases, such as the World Atlas of Language Structures \parencite{wals} and PHOIBLE \parencite{phoible}, that include information on the grammar, phonology, and geography of potentially up to thousands of languages. The existence of such databases has enabled linguists to conduct much more complex analyses of cross-linguistic typology, and their existence has consequently been pivotal for comparative and historical linguistics. One particular application of linguistic databases may be found in comparative phonology, especially when they may be used in attempting to identify potential extralinguistic factors that may act as predictors for linguistic characteristics. Questions about this would have been essentially impossible to answer prior to the development of these databases, but can now be easily researched by any linguist with access to a computer and a statistical computing software package. 

It is therefore unsurprising that within the last 20 years, a relatively wide body of research has appeared, essentially out of thin air, that attempts to diagnose the origins of various phonological traits. Different researchers have honed in on a variety of phonological features and predictors thereof — for example, \textcite{fought2004} identified a purported relationship between sonority (a metric of the frequency and intensity of CV syllables in a language) and climate, then \textcite{ember2007} expanded on this to also test whether sexual mores of the surrounding culture may impact the sonority of a language; another study from the same year identified a negative correlation between population size and overall phoneme inventory size \parencite{hay2007}. Distributional phonological typology has become a distinct enough field of linguistic exploration that it has even become a major research focus for some scholars, such Caleb Everett, an anthropologist at the University of Miami, who has coauthored multiple papers \parencite{everett2013,everett2015,everett2016} examining the relationship between various phonological features and extralinguistic environmental factors. Everett's work has been cited as a basis for further research by other scholars, such as \textcite{bentz2018} and \textcite{noelle2020}, even when branching out into distinct hypotheses and methods (as did these studies).

Although much interest has developed surrounding this type of research, not all of said interest is positive in nature, and significant ongoing controversy exists with respect to some conclusions that have been made \parencite{ladd2015}. In fact, Everett himself was embroiled in a minor firestorm within both scholarly literature and the linguistics blogosphere after the publication of a 2013 article, "Evidence for Direct Geographic Influences on Linguistic Sounds: The Case of Ejectives," wherein he proposed a causative effect between elevation and the development of ejective consonants on the basis of a rudimentary correlational analysis. One of the quickest responses to this came from \textcite{hammarstroem2013} in the form of an informal blog post — Hammarström performed an informal but more methodologically rigorous replication of Everett's study and found no significant correlation between elevation and ejectives, criticizing what he perceived to be data dredging and improper operationalizing of elevation on the part of Everett. \textcite{haynie2014,dediu2017} echo Hammarström's criticism of Everett's choice to operationalize elevation as a set of dichotomous variables, and \textcite{haynie2014} advised that more caution be taken in interpreting correlations observed between environmental factors and linguistic phenomena in general.

\textcite{urban2021} presented an even more detailed challenge to Everett's analysis in the form of an original research study, much more thorough than \textcite{hammarstroem2013}, in order to investigate the conclusions made by Everett (among several other goals which are not relevant to this particular discussion). In this study, they considered not only environment, but the possibility of phonological influences exerted by sociolinguistic isolation and diachronic language contact. Ultimately, they concluded that the last would generally be a much more likely proximate cause of the development of ejective consonants in high-elevation languages, although the \emph{ultimate} cause of the phenomenon remains unclear (and could still be, as Everett proposed, elevation).

A significant gap in the analyses of both \textcite{everett2013} and \textcite{urban2021} exists in that elevation is neither an independent measure from other climate factors nor a direct proxy --- that is, there are other traits of climate that relate to elevation, but not in a completely uniform manner (such as air temperature, seasonal fluctuation thereof, precipitation, etc.). Both studies measure the relationship between the presence of ejectives and elevation, but not any additional climatic factors. As a result, climatic traits carry the potential to act as confounding variables --- in the analyses these studies conduct, there is no way to know whether elevation or, e.g., precipitation is actually exerting influence on the prevalence of ejectives among world languages. This is a major issue that ought to be clarified before further research uncritically accepts \textcite{everett2013} as a foundation, as a number of studies have already done. In order to resolve this knowledge gap, I have conducted an analysis that incorporates not just elevation, but also key climate factors that could be true causes of the development of ejectives (average annual temperature, annual temperature fluctuation, and precipitation). I have also taken into account geography and vertical inheritance, in accordance with the criticisms of \textcite{everett2013} levied by \textcite{hammarstroem2013}.

\section{Materials and methods}

In keeping with \textcite{urban2021}, I used the PHOIBLE phonological database \parencite{phoible} in order to collect ejective and overall phoneme counts from world languages. At the time of writing, PHOIBLE contains information on 3,020 distinct phonemic inventories, of which 2,020 were ultimately included in my analysis (the others being duplicates or lacking sufficient data), from 2,186 languages across the world. The central coordinates at which each language is spoken were obtained from Glottolog 4.6 \parencite{glottolog}, and these were mapped onto NOAA's ETOPO global topographical/bathymetric elevation model \parencite{etopo} in order to approximate the elevation at which the languages would be spoken. Key climate data for each language (average annual temperature, average annual temperature range, and average annual precipitation) were obtained by projecting Glottolog coordinates onto WorldClim BIO's \parencite{worldclim} maps, which provide averaged data from between 1970 and 2000 for these variables, among others.

Statistical analysis was performed in R version 4.2.1 \parencite{R} using the packages \texttt{dplyr} \parencite{dplyr}, \texttt{car} \parencite{car}, and \texttt{raster} \parencite{raster}, with figures generated using \texttt{ggplot2} \parencite{ggplot2}, and \texttt{ggpubr} \parencite{ggpubr}. The presence of ejectives was operationalized in three ways: a dichotomous variable indicating presence/absence (as in \cite{everett2013}), an integer representing the absolute number of ejective phonemes in a given language, and a ratio of ejectives to all obstruents. Mann-Whitney-Wilcoxon $U$ tests were performed in order to determine whether elevation and climate variables differed between languages with and without ejectives. Type III analysis of covariance (ANCOVA)-like tests were performed in order to determine how vertical inheritance, horizontal phoneme borrowing, elevation, and climate factors impact all three measures of ejective prevalence. The approach taken to analyzing the binary dependent variable corresponding to ejective presence/absence differs from that taken by \textcite{urban2021}, who performed a logistic mixed-effects regression analysis; I was unable to take such an approach, although it would be more robust, as I do not have access to powerful enough computer hardware to meet the needs of such a computationally intensive analysis. 

\section{Results}

\begin{figure}
	\includegraphics[width=\linewidth]{plots/boxplots.png}
	\caption{Box plots illustrating the environmental differences between languages with and without ejectives. With respect to each coordinate point at which a language is spoken, (a) shows its approximate elevation in meters, (b) shows its average annual temperature, (c) shows the average annual temperature range (max - min), and (d) shows the average annual precipitation in millimeters.}
	\label{fig:boxplots}
\end{figure}

Highly significant differences (Mann-Whitney-Wilcoxon $U$, $p << 0.01$) in means for elevation and all three climate variables were found between languages with and without ejectives (see Figure \ref{fig:boxplots}). Many outliers in the set of languages lacking ejectives were present for all measurements. Significant negative correlations existed between elevation and temperature, temperature range, and precipitation, although there was significant spread in the data (see Figure \ref{fig:scatterplots}). These were taken into account for the ANCOVA-like tests.

\begin{figure}
	\includegraphics[width=\linewidth]{plots/scatters.png}
	\caption{Scatter plots of elevation and the three climate variables. Each point represents an individual language. Point colors represent primary language family. Open circles represent languages without ejectives and filled circles represent languages with ejectives. (a)-(c): entire dataset, (d)-(f): languages with ejectives only, (g)-(i): languages without ejectives only.}
	\label{fig:scatterplots}
\end{figure}

\begin{figure}
	\centering
	\begin{tabular}{ccccccc}
		\toprule
		\textbf{Predictor} & \textbf{Number} & & \textbf{Ratio} & & \textbf{Presence} & \\
		\midrule
		\rowcolor{yellow} Language family & $3.3 \times 10^{-61}$ & *** & $2.9 \times 10^{-55}$ & *** & $4.6 \times 10^{-55}$ & *** \\
		Macroarea & $0.46$ & & $0.03$ & * & $0.045$ & * \\
		Elevation & $0.49$ & & $0.29$ & & $0.42$ & \\
		Avg. annual temp. & $0.83$ & & $0.63$ & & $0.96$ & \\
		Annual temp. range & $0.17$ & & $0.32$ & & $0.32$ & \\
		Avg. annual precip. & $0.66$ & & $0.22$ & & $0.50$ & \\
		\rowcolor{yellow} Language family $\times$ Elevation & $2.1 \times 10^{-37}$ & *** & $1.0 \times 10^{-15}$ & *** & $4.5 \times 10^{-16}$ & *** \\
		Macroarea $\times$ Elevation & $.000604$ & *** & $0.087$ & & $0.10$ & \\
		Elevation $\times$ Avg. annual temp. & $0.066$ & & $0.012$ & * & $0.029$ & * \\
		Elevation $\times$ Annual temp. range & $0.041$ & * & $0.17$ & & $0.12$ & \\
		\rowcolor{yellow} Elevation $\times$ Avg. annual precip. & $0.0066$ & ** & $0.00076$ & *** & $0.0060$ & ** \\
		Avg. annual temp. $\times$ Avg. annual precip. & $0.33$ & & $0.064$ & & $0.19$ & \\
		Annual temp. range $\times$ Avg. annual precip. & $0.18$ & & $0.36$ & & $0.21$ & \\
		Avg. annual temp. $\times$ Annual temp. range & $0.86$ & & $0.54$ & & $0.59$ & \\
		\bottomrule
	\end{tabular}
	\caption{$p$-values for predictors of all three response variables (number of ejectives, ratio of ejectives to all obstruents, presence/absence of ejectives). Highlighted rows signify predictors that were significant for all three dependent variables. Number of asterisks indicates significance (one asterisk: $p < 0.05$; two: $p < 0.01$; three: $p < 0.001$). Rows with factor labels ``Predictor $\times$ Predictor'' represent interaction effects.}
	\label{fig:pvals}
\end{figure}

ANCOVA-like tests revealed highly significant correlations ($p << 0.01$) between primary language family and all three measures of ejective prevalence (absolute count, ratio to overall obstruents, and presence/absence), when controlling for climate factors. Macroarea was somewhat significantly correlated with ratio of ejectives to obstruents and presence/absence of ejectives, but exhibited no significant association with absolute number of ejectives. None of the environmental factors alone, including elevation, were significantly associated with any of the response variables.

The effect of interactions among elevation and the climate variables was also included in the models in order to control for the observed colinearity of these variables. Despite the lack of significance of average annual precipitation alone, it had a significant to highly significant ($p < 0.01$) interaction effect with elevation for all response variables. Elevation had a somewhat significant ($p < 0.05$) interaction with average annual temperature for ratio and presence/absence; this interaction bordered on significant ($p = 0.066$) for absolute number of ejectives. No other interactions were consistently significant (see \ref{fig:pvals}). 

\section{Discussion}
There are significant limitations that merit consideration when evaluating these results, but they appear potentially ameliorating for \textcite{everett2013} and broadly consistent with the conclusions made by \textcite{urban2021}. Although elevation alone does not exhibit a significant relationship with any of the three measures of ejective prevalence in my data, there was a significant interaction effect between language family and elevation for all measures, which is more meaningful than elevation alone in this type of analysis. This can be interpreted as meaning that the strength of the effect exerted by vertical inheritance on the prevalence of ejectives in a language is dependent on the elevation at which the language is spoken, which is essentially what would be predicted by \textcite{everett2013}. This does not necessarily have any bearing on the explanations \textcite{everett2013} presents for this trend, but at the very least, the trend does seem to be real, contrary to what \textcite{hammarstroem2013} reported.

The horizontal transmission hypothesis put forward by \textcite{urban2021} also appears to be somewhat consistent with these data. Although macroareas are an extremely general means of measuring diachronic language contact and it would be impossible to make a strong conclusion on the basis of this analysis (or any analysis, since the long-term phonological and geographical histories of most languages can only be inferred) about the impact of language contact, the significant relationship between macroarea and the two more robust measures of ejectives, as well as the marginally significant interactions between elevation and macroarea, point to the possibility that diachronic language contact \emph{may} increase the probability of adoption of ejectives, as \textcite{urban2021} postulate.\footnote{According to \textcite{urban2021}, this is essentially certain to have happened in a few cases, such as that of Ossetian, which is an Indo-European language (which almost always lack ejectives) which likely adopted ejectives after extensively interacting with the indigenous Caucasian peoples (the languages of whom are characterized by their liberal use of ejectives). It is simply uncertain whether this is a broader trend.} However, one must take into account that there was not a significant relationship between macroarea and the absolute number of ejectives in a given language, and the $p$-values for the other two response variables were relatively large --- the case is certainly not yet closed, so to speak. 

Although the interaction effect between language family and elevation does not support \textcite{everett2013}'s explanations for the association between elevation and ejectives, one of his explanations \emph{is} potentially supported by the interaction effect between elevation and precipitation. This effect is consistent with \textcite{everett2013}'s secondary explanation for the observed trend between ejectives and elevation, which is that articulating more ejective consonants and fewer pulmonic consonants could act as a water-saving measure in dry alpine environments. This was merely speculation on his part, as he did not test any measure of humidity (nor did I, since it is not widely measured or provided in climate data and is not trivial to derive from other climate statistics, per \textcite{eccel2012}, but amount of precipitation may serve as a very rough proxy for susceptibility to dehydration for the purposes of this research). This interaction effect and the results of the Mann-Whitney-Wilcoxon $U$ test for precipitation suggest that ejectives may be particularly likely to develop in dry high-elevation environments, where water-saving measures would presumably be especially necessary.

Although average annual temperature did not exhibit a significant relationship with any measure of ejective prevalence, its interaction effect with elevation was within the $0.01 < p < 0.1$ range for all measures --- although this is a somewhat uncertain relationship, this could indicate that colder high-elevation regions are also more likely to exhibit ejectives. If the aforementioned hypothesis about ejectives preserving water is correct, it could potentially explain this observed correlation --- cold air is generally drier than warm air \parencite{koskela2007}, so water-saving measures would also be more useful in cold air.

As I mentioned above, there are major limitations to the statistical analysis that I performed. First of all, languages' geographic ranges are represented in Glottolog as single coordinate pairs. This is ultimately the only tenable approach for storing geographical data about languages because they are always in flux and may have very complex boundaries, and it is likely reasonably accurate for most languages, which tend to be spoken in small geographical ranges \parencite{hua2019}; however, it does fail to capture the geographical diversity of languages spoken across broader swathes of land, such as Arabic, English, Spanish, French, Russian, Portuguese, etc. Elevation and climate data for languages like these would be restricted to only a single, effectively arbitrary point in space, which may not be representative of the environment in which the majority of speakers live. 

Secondly, there exist additional possible predictors for ejective prevalence that simply could not be incorporated in my analysis. \textcite{nichols2014}, for example, proposed that sociolinguistic isolation could lead to the development of ``more complex'' segments like ejectives and uvulars; my analysis does not account for sociolinguistic isolation, which is difficult to quantify and certainly not easily measurable on the scale at which this analysis was conducted. \textcite{urban2021} found some evidence suggesting that the isolation hypothesis may be inaccurate, but they did not directly measure sociolinguistic isolation either, and I do not find their analysis of it sufficiently robust to dismiss it wholesale as a possible factor. Therefore, I continue to consider my analysis's omission of it to be a reasonable cause for caution. Likewise, I could not directly analyze the effect of diachronic language contact, which \textcite{urban2021} suggested as a primary factor leading to ejectives; the manner in which they analyzed this possibility was also rather indirect and specific to the situations of Indo-European and Sino-Tibetan, since as previously mentioned, it is not possible to chronicle the true history of language contact for most languages. My inability to measure this directly could lead to an overestimate of the effects of climate on ejective prevalence compared to diachronic contact and phoneme borrowing, which could be cause for concern. 

A final concern is that there exists a major counterexample to every single trend I have identified that seems quite difficult to explain: despite the Tibetan plateau being extremely high-elevation (containing the highest point in the world), generally quite dry, and very cold, languages spoken there do not have ejectives. \textcite{everett2013} also made note of this in his own discussion, though only in passing; he didn't present a clear justification for the Tibetan plateau's exceptional nature. Without discarding all of the conclusions drawn previously, the only salient explanations for this would appear to be (1) random chance (which seems unlikely given that several dozen languages spoken in this region were included) or (2) that the fact that inhabitants of the Tibetan plateau exhibit distinct biological adaptations to hypoxia might render behavioral/linguistic adaptations unnecessary \parencite{beall2000,petousi2014}. The latter explanation would appear sensible on its face, but it is complicated by the fact that Andeans exhibit similar biological adaptations \parencite{beall2000}, yet have languages with ejectives. One could point out that there do exist distinctions in the precise adaptations exhibited by Andeans and Tibetan plateau residents \parencite{beall2002}, but these distinctions seem unlikely to produce such a disparity in the frequency of ejectives in languages spoken by these locations' respective indigenous peoples. Further research could focus on attempting to elucidate why the linguistic trends I and previous researchers have identified do not hold up in the Tibetan plateau.

Despite the limitations inherent to this research design, my hope is that these data may help to clarify and illuminate additional aspects of the debate over environmental impacts on ejectives. Based on a synthesis of my results, those of \textcite{everett2013}, and those of \textcite{urban2021}, it appears that ejectives may be more likely to initially develop when the speakers live at high elevations, especially in areas with colder temperatures and/or less precipitation, and may also be more likely to disappear if speakers live at low elevations. This may be explained in part as a function of water loss: articulating ejectives is less expensive than articulating equivalent pulmonic consonants, so in environments where water loss is of greater concern, ejectives may be more felicitous in general. Ejectives may additionally be propagated after initial genesis through phonological exchange with other languages in prolonged and/or close contact. More research is warranted in order to determine the extent to which other factors may also impact the likelihood of the development of ejectives, and what has been uncovered so far may be only a piece of the puzzle.

\section{Acknowledgments}

Much of the code that I used to retrieve and sort data from PHOIBLE and Glottolog was derived from that of Matthias Urban and Steven Moran, who have graciously made the source code for \textcite{urban2021} publicly available on a GitHub repository, accessible at https://github.com/urban-m/elev. I am indebted to both researchers; their commitment to open science made this analysis much easier to conduct. I have likewise made the source code I used to conduct my analysis available at the following URL: https://www.github.com/lyndsayricks/wrtg3012

\pagebreak

\printbibliography

\pagebreak

\section{Annotated bibliography}
{\parindent0pt Note: I have intentionally arranged the entries in this bibliography non-alphabetically in favor of providing them in an order that is more coherent and connected.}\singlespacing
\begin{itemize}
\item \textbf{\fullcite{everett2013}}

I actually summarized and analyzed this article in much more depth in my critical analysis paper, but I will repeat the main points here.

\textcite{everett2013} builds on previous research pertaining to environmental influences on linguistic typology, examining the distributional typology of ejective consonants, which had not been previously researched. He performed correlational and observational tests on the elevation at which languages with and without ejectives were spoken as well as the \emph{proximity} to a high elevation region of languages with and without ejectives. He found that there was a distinctly higher chance of a language having ejectives if it were spoken at a higher elevation, proposing two potential explanations for this phenomenon: that ejectives occur more often at higher elevations because the lower air pressure associated with high-elevation regions makes them easier to articulate, and/or that they occur more often at higher elevations because higher elevations are associated with drier air and articulating ejectives requires less water than articulating equivalent pulmonic obstruents. 

This study was quite controversial and its methodology and reasoning were problematic in several ways. First of all, Everett made the \emph{a priori} assumption that a relationship between the presence of ejectives and elevation would indicate that elevation \emph{causes} ejectives to be more likely to appear. This is problematic because Everett fails to examine any possible alternative explanations, despite the fact that elevation covaries with other environmental and social factors that could potentially be the ultimate cause of this relationship (as I demonstrated in this paper) --- in other words, he makes the basic error of failing to control for confounding factors. Additionally, his measurements of elevation and high-elevation regions were problematic because they were primarily operationalized as binary variables (e.g., high elevation vs. low elevation) rather than continuous integers, which would be more sensible. His speculative explanations were therefore largely unjustified by his data.

\textcite{everett2013} constitutes one of the two primary bases for this study. Despite the flaws with the way in which he addressed it, his research question was interesting and does merit attention; the goal of this study was more or less to answer this question in a more careful and statistically robust manner. The foundation of my paper relies on a synthesis of both Everett's theoretical framework and \textcite{urban2021}'s methods, along with a few of my own original additions.

\item \textbf{\fullcite{hammarstroem2013}}

\textcite{hammarstroem2013} outlines an independent statistical analysis he performed in direct response to \textcite{everett2013}, specifically using languages that were isolated from one another both in terms of geography and inheritance (never spoken near one another and not originating from the same language family). He states that he found no significant relationship between elevation and the presence of ejectives when using a sample of 19 such languages, proceeding to criticize Everett's methodology on a few additional grounds: Hammarström argues that Everett has defined ``major regions'' of high elevation in a subjective and erroneous manner, that he failed to adequately explain how and why he clustered non-ejective languages into groups of 10, that he arbitrarily selected thresholds of 200 and 500 km from high-elevation regions when performing tests on elevation proximity (and may have engaged in data dredging to get to this point), and that his choice of linguistic database (i.e., WALS) was inappropriate and he should have used a more representative database like PHOIBLE \parencite{phoible}.

I must emphasize that \textcite{hammarstroem2013} is \emph{not} a peer-reviewed source. It is actually just a simple blog post, albeit one written by a competent and qualified linguist (Hammarström is a professor of linguistics and philology at Uppsala University and the lead developer of Glottolog, a major database including data about languages' geography) and published on a relatively high-quality blog. Hammarström's analysis of only 19 languages is obviously not necessarily statistically meaningful and I think that it would be inappropriate to wholly discard the explanations presented by \textcite{everett2013} on the basis of that analysis alone. However, I view Hammarström's criticisms of \textcite{everett2013}'s methodology to be perfectly reasonable, especially since he has quite a bit of authority on this subject. \textcite{urban2021} clearly took into account some of the methodological considerations presented by \textcite{hammarstroem2013}, and I have done the same. 

\item \textbf{\fullcite{urban2021}}

\textcite{urban2021} constitutes the second pillar of the original research outlined here. They expanded on \textcite{everett2013}'s research question by using more robust statistics to answer the same question, as well as by examining uvulars, which are distributed across languages similarly to ejectives. In addition to looking at elevation, they also researched the likelihood that horizontal transmission of phonemes could be a more likely cause of the proliferation of ejectives in coterminous language families by performing statistical analysis on two example language families (Indo-European and Sino-Tibetan), concluding that this would be a likely proximate cause for the development of ejectives in languages like Ossetian and Eastern Armenian, and perhaps in older language groups as well.

\textcite{urban2021} represented a major step up compared to \textcite{everett2013} in terms of the caliber of its analysis, but left some remaining questions to be answered. Like \textcite{everett2013}, they did not address potential confounding environmental factors with respect to elevation, and their analysis of the role of language contact could only be applied to the languages that they studied --- it could not be extrapolated to make wider conclusions about language in general. 

The statistical basis of \textcite{urban2021} is more robust, particularly because they, unlike \textcite{everett2013}, did not reduce elevation to a binary variable. They also used a better database that was actually intended for typological research, a step which I replicated here. I retained many of the same aims in my statistical analysis, simply expanding it to cover more variables than theirs, and in fact was able to use some of their code for data extraction.

\item \textbf{\fullcite{ladd2015}}

\textcite{ladd2015} discuss a number of linguistic typological studies, among them \textcite{everett2013}. They identify methodological and logical concerns in some studies, focusing especially on concerns about the extent to which individual languages actually represent individual data points in light of historical contact and relatedness. They discuss controversies that have developed in relation to many claims on the subject of linguistic typology, ultimately contending that despite the flaws that do exist, the field is one that does warrant research and that reactions to linguistic typological resarch are often knee-jerk and should be more carefully considered. 

As \textcite{ladd2015} do not perform any original research of their own (the paper is a literature review), there is less to critically evaluate than in \textcite{everett2013} or \textcite{urban2021}. Their reasoning and original conclusions, however, are carefully and precisely made. It should be noted that two of the authors have previously worked with linguistic typology and might therefore be biased towards a positive conclusion in its direction, but it is clear that they have been sufficiently critical of the field.

With respect to the specific relationship that my paper concerns (ejectives and elevation), \textcite{ladd2015} said relatively little. They briefly reviewed \textcite{everett2013}, using it as an example of how correlation should not be misinterpreted as causation, but also clarifying that experimental data would be impossible to obtain on the subject, because it would not be possible to organically reproduce the process of language evolution in experimentally controlled environmental niches. This paper is relevant to me primarily because of this part, but its overall conclusions about the veracity (or lack thereof) of linguistic typology also inform my research aims and the way in which I interpreted my data.

\item \textbf{\fullcite{haynie2014}}

The aims of \textcite{haynie2014} overlap somewhat with those of \textcite{ladd2015}, and like \textcite{ladd2015}, the paper is simply a literature review. The two differ, however, in that \textcite{haynie2014} is more specifically focused on geospatial relationships and is focused not just on phonology and syntax, but also dialectology and language evolution in general (whereas \cite{ladd2015} deal with geographical, sociolinguistic, and biological proposals explaining trends in phonology/syntax). \textcite{haynie2014} contains three main sections (respectively concerning dialectology, inference of linguistic ``homelands,'' and geographic studies of linguistic diversity) along with an introduction and a conclusion; only the third main section is of major relevance to my paper. In this section, she reviews \textcite{everett2013} and several other papers concerning environmental influence on phonology, such as \textcite{fought2004}. Echoing \textcite{hammarstroem2013}, she criticizes the way in which some researchers, including Everett, have haphazardly binned continuous variables into ordinal variables for the sake of analysis (although she words this somewhat more gently than I did here). Uniquely, Haynie criticizes determinism in drawing conclusions from the type of hazy data obtained from typological studies, which she again implicates \textcite{everett2013} in. Nonetheless, she ultimately expresses similar views to \textcite{ladd2015} with regards to the potential value of research on geographical variation in linguistic traits --- she writes that such work is ``vital to developing a better understanding of language variation and consequently to our ability to reconstruct linguistic prehistory'' \parencite[352]{haynie2014}.

Again, as this is merely a literature review, there is less to evaluate here than there would be in a research paper. The conclusions reached by Haynie are generally quite measured and clearly supported based on a review of a reasonable variety of literature, albeit not nearly as much as \textcite{ladd2015} considered --- this is, however, understandable, considering that this subject constituted only a section of \textcite{haynie2014}, whereas it was essentially the entire focus of \textcite{ladd2015}. \textcite{haynie2014} is useful to me because it makes a point about \textcite{everett2013} that \textcite{ladd2015} did not, namely that Everett's conclusions exhibit a severely deterministic outlook on his part, which may or may not be justified. I took this into account heavily, considering that my research is essentially intended to reevaluate these exact conclusions.

\item \textbf{\fullcite{dediu2017}}

\textcite{dediu2017} is another review paper overlapping somewhat with the previous two, but it has more of a biological focus, discussing how genes and anatomy figure into language evolution. Like in \textcite{haynie2014}, they discuss phonology and environment in only one, relatively small section (\S 2.3), which is the one of primary relevance to me. Here, they discuss \textcite{everett2013,everett2015,everett2016} along with another study on non-human animal communication; with respect to \textcite{everett2013}, they echo \textcite{haynie2014}'s concerns about the way that altitudes were grouped, and they purport that ``it is not entirely clear how a lower atmospheric pressure might actually favor the production of ejectives'' \parencite[3]{dediu2017}, though they fail to elaborate on this point.

The points made by \textcite{dediu2017} with respect to the issue under investigation could stand to be fleshed out somewhat, but the caliber of the analysis in the article is perfectly acceptable. \textcite{dediu2017} helpfully illustrates another viewpoint in the debate over \textcite{everett2013} and is useful because it exhibits various types of distinct research questions about linguistic typology that are currently being pursued.

\item \textbf{\fullcite{fought2004}}

\textcite{fought2004}, unlike the previous three articles reviewed, actually details original research performed by the authors on linguistic typology, although it does not relate to ejectives. Instead, the authors research how climate variables relate to ``sonority,'' a calculated scale of acoustic ``power'' of the average syllable, in a given language. They used a sample of only 93 languages for similar reasons to those articulated by \textcite{hammarstroem2013}, avoiding spatio-cultural cross-contamination. They conducted statistical analysis for sonority with respect to climate type (modeled as a 5-level ordinal variable), finding that increasing sonority was associated with warmer climates.

The analysis conducted by \textcite{fought2004} was problematic in a few ways --- \textcite{haynie2014} expressed concerns with their unnecessarily modelling a continuous variable like climate as an ordinal variable (which seems to be a common problem in linguistic typological research, considering that the same issue appeared in \textcite{everett2013}, but with elevation rather than climate). Additionally, the scores that they obtained for acoustic power, which were used to calculate sonority scores, were calculated based on American English, not the individual languages being researched. Their sample size was also less than ideal.

Because \textcite{fought2004} do not discuss ejectives, their research is only relevant for the fact that they use climate as a predictor, as I did in this paper. As pointed out in the previous paragraph, the way in which they did so is obviously problematic and has been explicitly criticized in the current scholarly literature. Therefore, this paper serves as an example of what \emph{not} to do in my analysis (as well as a means of illustrating the current state of research on environmental linguistic typology). 

\item \textbf{\fullcite{ember2007}}

\textcite{ember2007} built on \textcite{fought2004}, also focusing on sonority and using the same data. Somewhat like my own research, they wanted to address the same subject, but using more predictor variables. They selected terrain roughness, plant cover, and sexual permissiveness of the society associated with a given language, and used only the core 60 data samples from \textcite{fought2004}. They concluded that the addition of terrain roughness and plant cover increased the predictive value of ``econiche'' as it pertains to sonority, and that sexual permissiveness was also a viable indicator of sonority.

\textcite{ember2007} exhibits even more methodological problems than \textcite{fought2004}, and might in fact be the most problematic of any of the papers I cited in this report. The apparently ubiquitous issues with operationalization of environmental and social variables in this type of research appear once again here: each independent variable except for the number of cold/warm months was modelled as ordinal rather than continuous, even though all of them could conceivably be operationalized as continuous variables. The most egregious offender here was ``sexual permissiveness,'' which is purportedly just a measure of the frequency of premarital and extramarital sex --- there is no cogent justification for making this variable ordinal. The concerns related to sample size and the measure of sonority from \textcite{fought2004} persist here, since the same data (except with even fewer samples) was used. Again, this study simply serves as an example of what not to do, and as a means of introducing the reader to this subfield of linguistics.
  
\end{itemize}

\end{document}
