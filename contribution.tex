\documentclass{article}
\usepackage[style=apa, backend=biber]{biblatex}
\addbibresource{contribution.bib}
\usepackage{graphicx}
\usepackage{setspace}
\usepackage[margin=1in]{geometry}
\usepackage{fancyhdr}
\pagestyle{fancy}
\fancyhf{}
\renewcommand{\headrulewidth}{0pt}
\fancyhead[R]{\thepage}
\doublespacing

\begin{document}
\begin{titlepage}
	\centering
	\vspace*{0.2\textheight}
	\textbf{An expanded glance at environmental influences on ejectives in world languages}\\
	Lyndsay Ricks\\
	Department of Linguistics, University of Utah\\
	WRTG3012: Writing in the Social Sciences\\
	Professor Nona Brown\\
	November 28, 2022\\
	DRAFT
\end{titlepage}
\section*{Abstract}

Previous studies have reported on an association between the likelihood of a given language having phonemic ejective consonants and the elevation at which it is spoken, but have omitted consideration of additional climatic factors. This study expands on previous research by measuring and controlling for three key climate features associated with elevation (average annual temperature, temperature range, and precipitation), and by using measures of not only the presence of ejectives, but also their prevalence. The results of this analysis indicated a significant relationship between elevation and ejectives, both in terms of presence and prevalence, but also an association between ejectives and precipitation, geographical area, and interaction effects between climate variables. Implications for future linguistic typological research are discussed. \\

{\parindent0pt Keywords: \emph{phonology, climate, linguistic typology, ejectives, elevation, climate, linguistic geography, ecolinguistics}}

\pagebreak

\section{Introduction}
In recent years, rapid technological development has led to the proliferation of advanced databases, such as the World Atlas of Language Structures (\cite{wals}) and PHOIBLE (\cite{phoible}), that include information on the grammar, phonology, and geography of potentially up to thousands of languages. The existence of such databases, naturally, has enabled linguists to conduct much more complex analyses of cross-linguistic typology, and their existence has consequently been pivotal for comparative and historical linguistics. One particular application of linguistic databases may be found in comparative phonology, especially when they may be used in attempting to identify potential extralinguistic factors that may act as predictors for linguistic characteristics. Questions about this would have been essentially impossible to answer prior to the development of these databases, but can now be easily researched by any linguist with access to a computer and a statistical computing software package. 

It is therefore unsurprising that within the last 20 years, a relatively wide body of research has appeared attempting to diagnose the origins of various phonological traits, even though virtually no such research had existed before. Different researchers have honed in on a variety of phonological features and predictors thereof — for example, one 2007 paper expanded on the methods of \textcite{fought2004}, which identified a purported relationship between sonority and climate, to also test whether sexual mores of the surrounding culture may impact the sonority of a language (\cite{ember2007}); another from the same year identified a negative correlation between population size and overall phoneme inventory size (\cite{hay2007}). Distributional phonological typology has become a distinct enough field of linguistic exploration that it has even become a major research focus for some scholars, such Caleb Everett, an anthropologist at the University of Miami, who has coauthored multiple papers (e.g., \cite{everett2013}; \cite{everett2015}; \cite{everett2016}) examining the relationship between various phonological features and extralinguistic environmental factors. Much of Everett's work has been cited as a basis for further research by other scholars, such as \textcite{bentz2018}, \textcite{dediu2017}, and \textcite{noelle2020}, even when branching out into distinct hypotheses and methods (as did these studies).

Although much interest has developed surrounding this type of research, not all of said interest is positive in nature, and significant ongoing controversy exists with respect to some conclusions that have been made (\cite{ladd2015}). In fact, Everett himself was embroiled in a minor firestorm within both scholarly literature and the linguistics blogosphere after the publication of a 2013 article, "Evidence for Direct Geographic Influences on Linguistic Sounds: The Case of Ejectives," wherein he proposed a causative effect between elevation and the development of ejective consonants on the basis of a rudimentary correlational analysis. One of the quickest responses to this came from \cite{hammarstroem2013} in the form of an informal blog post — Hammarström, a Swedish linguist, performed an informal but more methodologically rigorous replication of Everett's study and found no significant correlation between elevation and ejectives, criticizing what he perceived to be data dredging and improper operationalizing of elevation on the part of Everett. \textcite{haynie2014} echoes Hammarström's criticism of Everett's choice to operationalize elevation as a set of dichotomous variables, also advising that more caution be taken in interpreting correlations observed between environmental factors and linguistic phenomena.

Most relevantly to this paper, \textcite{urban2021} have presented what is perhaps the most detailed challenge to Everett's analysis yet. Urban and Moran performed a more elaborate study, which was much more consistent with norms of scientific statistical analysis than was Everett's, in order to investigate the conclusions made by Everett (among several other goals which are not relevant to this particular discussion); in this study, they considered not only environment, but the possibility of phonological exchange through diachronic language contact. Ultimately, they concluded that the latter would generally be a much more likely proximate cause of the development of ejective consonants in high-elevation languages, but the ultimate cause of this remains unclear (and could still be, as Everett proposed, elevation).

A significant gap in the analyses of both \textcite{everett2013} and \textcite{urban2021} exists in that elevation is neither an independent measure from other climate factors nor a direct proxy --- that is, there are other traits of climate that relate to elevation, but not in a completely uniform manner (such as air temperature, seasonal fluctuation, precipitation, etc.). Both studies measure the relationship between the presence of ejectives and elevation, but not any additional climatic factors. As a result, climatic traits could act as confounding variables --- in the analyses these studies conduct, there is no way to know whether elevation or, e.g., precipitation is actually exerting influence on the prevalence of ejectives among world languages. In order to resolve this knowledge gap, I have conducted an analysis that incorporates not just elevation, but also key climate factors that could be true causes of the development of ejectives.

\section{Materials and methods}

In keeping with \textcite{urban2021}, I used the PHOIBLE phonological database (\cite{phoible}) in order to collect ejective and overall phoneme counts from world languages. At the time of writing, PHOIBLE contains information on 3,020 distinct phonemic inventories, of which 2,020 were ultimately included in my analysis (the others being duplicates or lacking sufficient data), from 2,186 languages across the world. The central coordinates at which each language is spoken were obtained from Glottolog 4.6 (\cite{glottolog}), and these were mapped onto NOAA's ETOPO global topographical/bathymetric elevation model (\cite{etopo}) in order to approximate the elevation at which the languages would be spoken. Key climate data for each language (average annual temperature, average annual temperature range, and average annual precipitation) were obtained by projecting Glottolog coordinates onto WorldClim BIO's (\cite{worldclim}) maps, which provide data from between 1970 and 2000.

Statistical analysis was performed in R version 4.2.1 (\cite{R}) using the packages \texttt{dplyr} (\cite{dplyr}) and \texttt{raster} (\cite{raster}), with figures generated using \texttt{ggplot2} (\cite{ggplot2}), and \texttt{ggpubr} (\cite{ggpubr}). The presence of ejectives was operationalized in three ways: a dichotomous variable indicating presence/absence (as in \cite{everett2013}), an integer representing the absolute number of ejective phonemes in a given language, and a ratio of ejectives to non-ejective obstruents. Mann-Whitney-Wilcoxon U tests were performed in order to determine whether elevation and climate variables differed between languages with and without ejectives. Analysis of covariance (ANCOVA) tests were performed in order to determine how vertical inheritance, horizontal phoneme borrowing, elevation, and climate factors impact absolute ejective counts and ratios, as well as whether a language has ejectives in general. The approach taken to analyzing the third variable differs from that used by \textcite{urban2021}, which was to perform a logistic mixed-effects regression analysis; I was unable to use such an approach, although it would be more robust, as I do not have access to powerful enough computer hardware to meet the needs of such a computationally intensive analysis. 

\section{Results}

\begin{figure}
	\includegraphics[width=\linewidth]{plots/boxplots.png}
	\caption{Box plots illustrating the environmental differences between languages with and without ejectives. With respect to each region at which a language is spoken,  (a) shows its approximate elevation in meters, (b) shows its average annual temperature, (c) shows the average annual temperature range (max - min), and (d) shows the average annual precipitation in millimeters.}
	\label{fig:boxplots}
\end{figure}

Highly significant differences (Mann-Whitney-Wilcoxon $p << 0.01$) in means for elevation and all three climate variables were found between languages with and without ejectives (see Figure \ref{fig:boxplots}). Many outliers in the set of languages lacking ejectives were present for all measurements. Significant negative correlations existed between elevation and temperature, temperature range, and precipitation, although there was significant spread in the data (see Figure \ref{fig:scatterplots}). These were taken into account in further data analysis.

\begin{figure}
	\includegraphics[width=\linewidth]{plots/scatters.png}
	\caption{Scatter plots of elevation and the three climate variables. Each point represents a language. Point colors represent primary language family. Open circles represent languages without ejectives and filled circles represent languages with ejectives. (a)-(c): entire dataset, (d)-(f): languages with ejectives only, (g)-(i): languages without ejectives only.}
	\label{fig:scatterplots}
\end{figure}

ANCOVA-like tests revealed highly significant correlations ($p << 0.01$) between primary language family (i.e., vertical inheritance), macroarea (representing horizontal borrowing), and elevation and all three measures of ejective prevalence (absolute count, ratio to overall obstruents, and presence/absence), when controlling for climate factors. Average precipitation alone was significantly ($ p < 0.01$) related to ratio and presence/absence, but only marginally so ($0.05 < p < 0.1$) to absolute count. Average annual temperature alone had a marginally significant relationship to absolute count and no significant relationship to the other two measures. Annual temperature range alone had no significant relationship with any measures of ejective prevalence. 

The effect of interactions among elevation and the climate variables was also included in the models in order to control for the observed colinearity of these variables. Despite the lack of significance of annual temperature range alone, it had a significant to highly significant ($p < 0.01$) interaction effect with elevation. Average precipitation had a moderately significant interaction effect with elevation in all cases ($0.005 < p < 0.05$). There was also a highly significant ($p << 0.01$) interaction effect between macroarea and elevation.

\section{Discussion}
There are significant limitations that merit consideration when evaluating these results, but they appear on their face to likely support a combination of the hypotheses put forward in \textcite{everett2013} and \textcite{urban2021}. The highly significant correlation between elevation and all three measures of ejectives even when controlling for alternative climatological and linguistic predictors suggests that Everett may be correct in his supposition that high-altitude living may predispose a language to developing ejectives. Macroareas are an extremely general means of measuring diachronic language contact and it would be impossible to make a strong conclusion on the basis of this analysis (or any analysis, since the long-term phonological and geographical histories of most languages can only be inferred) about the impact of language contact, but the significant relationship between macroarea and all three measures of ejectives points to the possibility that diachronic language contact \emph{may} increase the probability of adoption of ejectives, as \textcite{urban2021} postulate.\footnote{According to \textcite{urban2021}, this is essentially certain to have happened in a few cases, such as that of Ossetian, which is an Indo-European language (which almost always lack ejectives) which likely adopted ejectives after extensively interacting with the indigenous Caucasian peoples (the languages of whom are characterized by their liberal use of ejectives). It is simply uncertain whether this is a broader trend.}

In addition, the significance of both precipitation and its interaction effect with elevation indicates that precipitation may possibly play an ancillary role in the adoption of ejectives. This would be consistent with \textcite{everett2013}'s secondary explanation for the observed trend between ejectives and elevation, which was that articulating more ejective consonants and fewer pulmonic consonants could act as a water-saving measure in alpine environments, which tend to be drier. This was merely speculation on his part, as he did not test any measure of humidity (nor did I, since it is not widely measured or provided in climate data and is not trivial to derive from other climate statistics, per \cite{eccel2012}, but amount of precipitation may serve as a very rough proxy for susceptibility to dehydration for the purposes of this research). This statistical observation lends some amount of credence to that idea.

Although annual temperature range did not exhibit a significant relationship with any measure of ejectives, its interaction effect with elevation did for all measures --- this indicates that the extent to which elevation impacts ejective prevalence is dependent on how extreme the climate is in terms of temperature (with more extreme climates associated with increased ejective prevalence), or vice versa. This was not predicted by either \textcite{everett2013} or \textcite{urban2021}, and it is somewhat difficult to explain. If the aforementioned hypothesis about dehydration is correct, it is possible that this could play a role as well --- cold air is generally drier than warm air (\cite{koskela2007}), and broader annual temperature ranges inherently require that one part of the year must be cold (except in cases of exceptionally hot climates). It is possible that the development of ejectives in extreme climates could also be advantageous in terms of reducing water loss during the cold winters. However, this is entirely speculative, and it fails to explain why there would not be a relationship between temperature range alone and ejective prevalence; further research may be warranted in order to determine the nature of this association.

As I mentioned above, there are major limitations to the statistical analysis that I performed. First of all, languages' geographic ranges are represented in Glottolog as single coordinate pairs. This is ultimately the only tenable approach for storing geographical data about languages because they are always in flux and may have very complex boundaries, and it is likely reasonably accurate for most languages, which tend to be spoken in small geographical ranges (\cite{hua2019}); however, it does fail to capture the geographical diversity of languages spoken across broader swathes of land, such as Arabic, English, Spanish, French, Russian, Portuguese, etc. Elevation and climate data for languages like these would be restricted to only a single, effectively arbitrary point in space, which may not be representative of the environment in which the majority of speakers live. Secondly, there exist additional possible predictors for ejective prevalence that simply could not be incorporated in my analysis. \textcite{nichols2014}, for example, proposed that sociolinguistic isolation could lead to the development of ``more complex'' segments like ejectives and uvulars; my analysis does not account for sociolinguistic isolation, which is difficult to quantify and certainly not easily measurable on the scale at which this analysis was conducted. \textcite{urban2021} found some evidence suggesting that the isolation hypothesis may be inaccurate, but they did not directly measure sociolinguistic isolation, and I do not find their analysis of it sufficiently robust to dismiss it wholescale as a possible factor. Likewise, I could not directly analyze the effect of diachronic language contact, which \textcite{urban2021} suggested as a primary factor leading to ejectives; the manner in which they analyzed this possibility was also rather indirect and specific to the situations of Indo-European and Sino-Tibetan, since as previously mentioned, it is not possible to chronicle the true history of language contact for most languages. This could lead to an overestimate of the effects of climate on ejective prevalence compared to diachronic contact and phoneme borrowing. 

Despite the limitations inherent to this research design, it may help to clarify and illuminate additional aspects of the debate over environmental impacts on ejectives. Based on a synthesis of my results, those of \textcite{everett2013}, and those of \textcite{urban2021}, ejectives may be more likely to initially develop in the early stages of language divergence when the speakers live at high elevations, especially in areas with more extreme yearly temperature variation and/or less precipitation. This may be explained in part as a function of water loss: articulating ejectives is less expensive than articulating equivalent pulmonic consonants, so in environments where water loss is of greater concern, ejectives may be more likely to appear in general. Ejectives may additionally be propagated after initial genesis through phonological exchange with other languages in prolonged and/or close contact. More research is warranted in order to determine the extent to which other factors may also impact the likelihood of the development of ejectives, and what has been uncovered so far may be only a piece of the puzzle.

\section*{Appendix}

\subsection*{Acknowledgments}

Much of the code that I used to retrieve and sort data from PHOIBLE and Glottolog was derived from that of Matthias Urban and Steven Moran, who have graciously made the source code for \textcite{urban2021} publicly available on a GitHub repository, accessible at https://github.com/urban-m/elev. I am indebted to both researchers; their commitment to open science made this analysis much easier to conduct.

\subsection*{Source code}

I have likewise made the source code I used to conduct my analysis available at the following URL: https://www.github.com/lyndsayricks/wrtg3012

\pagebreak

\printbibliography
\end{document}